\documentclass[11pt]{beamer}
\usepackage[utf8]{inputenc}
\usepackage{lmodern} % Remarquer l’absence du package geometry
\usepackage{graphicx}
%\usetheme{default}
\usetheme{Boadilla}
%\usetheme{Darmstadt}
%\usetheme{Malmoe}
%\usetheme{Dresden}
%\usetheme{Frankfurt}
%\usetheme{Goettingen}
%\usetheme{Hannover}
%\usetheme{Ilmenau}
%\usetheme{AnnArbor}
%\usetheme{Antibes}
%\usetheme{Bergen}
%\usetheme{Berkeley}
%\usetheme{Berlin}
%\usetheme{Madrid}
%\usetheme{EastLansing}
%\usetheme{Luebeck}
%\usetheme{Darmstadt}
%\usetheme{Copenhagen}
%\usetheme{PaloAlto}
%\usetheme{Montpellier}
%\usetheme{Marburg}
%\usetheme{CambridgeUS}
%\usetheme{Pittsburgh}
\usepackage{xcolor}            % pour définir plus de couleurs 

\setbeamertemplate{caption}[numbered]
\usepackage{booktabs}
\usepackage{tabularx}
\usepackage{amsfonts}
\usepackage{amssymb}
\usepackage{bookmark}

%\usepackage{multicol}

\renewcommand{\figurename}{Graphique}
\renewcommand{\tablename}{Tableau}


%\def\couleur<#1>{\temporal<#1>{\color{gray!50}}{\color{red}}{\color{black}}}
\setbeamercovered{transparent}\AtBeginSection[]{\begin{frame}{Plan}\tableofcontents[currentsection]\end{frame}}
%}
  \titlegraphic{%
  \begin{picture}(0,0)
  %  \put(-125,15){\makebox(45,45)[rt]{\includegraphics[width=4.75cm]{MDC}}}
   %  \put(100,15){\makebox(45,35)[rt]{\includegraphics[width=1.55cm]{PNUD}}}
   % \put(145,10){\makebox(20,20)[rt]{\includegraphics[width=3.10cm]{uaclogo}}}
  \end{picture}}

\usepackage{caption}
\captionsetup{skip=0pt,belowskip=0pt}


\setbeamercolor{alerted text}{fg=blue}
\author [TASKFORCE]{\alert{TASKFORCE D'APPUI AU PROCESSUS }\inst{}}
\title[NLTPS 2060]{\large{\textbf{OUTILS OPERATIONNELS DU PROCESSUS}   }}


\date[ ]{\textbf{01 Février 2024}}


\begin{document} 

\begin{frame}
\titlepage
\end{frame}
 
\begin{frame}{Plan}
\tableofcontents    %[hideallsubsections]
\end{frame}

\section{Introduction}

\begin{frame}{Revue de la littérature}
\begin{itemize} [<+->]
\item  \small Aussitôt après son installation, la Taskforce s’est mise à l’œuvre en procédant entre autres à une revue de littérature sélective sur les approches méthodologiques en matière de réflexion prospective  \vfill 
\begin{enumerate} [<+->]
   \item  \tiny FUTURS AFRICAINS, Un guide pour les réflexions prospectives en Afrique, Futurs africains, Karthala et Futuribles, Dakar, 2001. \vfill
   \item \tiny GODET, M., Manuel de Prospective stratégique : Tome 1, Une indiscipline intellectuelle, Editions Dunod, Paris, 1997 \vfill
   \item  \tiny GODET, M., Manuel de Prospective stratégique : Tome 2, L’art et la méthode, Editions Dunod, Paris, 1997  \vfill
   \item  \tiny MECCAG-PDPE/PNUD, Etudes Nationales de Perspectives à Long Terme Bénin 2025 ALAFIA, Cotonou, 2000. \vfill
   \item \tiny NLTPS Benin 2025/PNUD/MECCAG-PDPE/CTP, Collecte des aspirations des populations béninoises, CTP/NLTPS, Cotonou, mars 2000 \vfill
   \item  \tiny NLTPS Benin 2025/PNUD/MECCAG-PDPE/CTP, Diagnostic stratégique de la Société Béninoise, Rapport de la phase 2, CTP/NLTPS, Cotonou, mars 2000 \vfill
  \item   \tiny NLTPS Benin 2025/PNUD/MECCAG-PDPE/CTP, Problématique du développement du Bénin, Rapport de la phase 1, CTP/NLTPS, Cotonou, mars 2000 \vfill
  \item   \tiny République de Guinée/Ministère du Plan et de la Coopération Internationale/PNUD/BAD, Vision 2040 pour une Guinée émergente et prospère, 2016. \vfill
\end{enumerate}
\end{itemize}
\end{frame}

\section{Note conceptuelle révisée}

\subsection{\tiny Place de l’identification des aspirations dans le processus}
\begin{frame}{Place de l’identification des aspirations dans le processus}
\begin{itemize} [<+->]
\item \textbf {Questions de terminologie}  \vfill 
\begin{itemize} [<+->]
   \item  \textbf{$\ll$ Collecte $\gg$} selon NLTPS Bénin 2025 et projet feuille de route Bénin 2060 versus \textbf{$\ll$ Identification $\gg$}  selon la Taskforce \vfill
   \item  \textbf{$\ll$ Identification $\gg$} pour refléter la réalité des activités à mener autour des aspirations \vfill
   \item  Les aspirations doivent être : \textbf{(i)} collectées; \textbf{(ii)} traitées; et \textbf{(iii)} analysées  \vfill
   \item  D’où \textbf{$\ll$ \alert{Identification} $\gg$} pour signifier la collecte, le traitement et l’analyse \vfill
\end{itemize}
\end{itemize}
\end{frame}

\begin{frame}{Place de l’identification des aspirations dans le processus}
\begin{itemize} [<+->]
\item \textbf {Questions de positionnement dans le processus}  \vfill 
\begin{itemize} [<+->]
   \item  Au début selon NLTPS Bénin 2025 et projet feuille de route Bénin 2060 : certains avantages mais aussi des inconvénients \vfill
   \item  Au stade de la construction des scénarios selon la Taskforce, pour les considérations ci-après : \vfill
   \begin{itemize} [<+->]
   \item  dichotomie entre projet d’avenir des populations et explorations des futurs possibles  \vfill
   \item  comme risques : \textbf {(i)}  suscitation d’aspirations générales et difficilement traduisibles ; \textbf {(ii)} prolongement de la durée du processus (lourdeur inhérente à l’opération d’identification) \vfill
   \end{itemize}
\end{itemize}
\end{itemize}
\end{frame}


\subsection{\tiny Choix des domaines et des thèmes des études prospectives}
\begin{frame}{Choix des domaines et des thèmes des études prospectives}

\begin{exampleblock}{  \textbf{ S’agissant des domaines } }
\begin{itemize} [<+->]
   \item   \textcolor{blue}{ESPECT} selon NLTPS Bénin 2025, \textcolor{blue}{ESEGP} selon la note conceptuelle initiale Benin 2060  \vfill
   \item    \textcolor{blue}{ESPECTISES} : Economie, Société, Politique, Environnement, Culture, Technologie, Infrastructures, Santé, Education, Sécurité \vfill
   \item \textbf{S’agissant des thématiques}, il est suggéré de retenir une formulation unique à savoir " Rétrospective, stratégies des acteurs et enjeux"
\end{itemize}
\end{exampleblock}
 
\begin{exampleblock}{ \textbf{ Avantages de cette suggestion } }
\begin{itemize} [<+->]
   \item   Orienter le contenu des rapports de sorte à faciliter leur exploitation aux étapes ultérieures du processus, notamment lors \vfill
    \begin{itemize} [<+->]
    \item  \small du \textbf{i.diagnostic stratégique},  de la \textbf{ii. construction des scénarios} et des \textbf{iii. choix stratégiques}.  \vfill
    \end{itemize}
\end{itemize}
\end{exampleblock}
\end{frame}


\begin{frame}{Choix des domaines et implications sur les groupes thématiques}
\small  Le choix des domaines \textcolor{blue}{\textbf{ ESPECTISES}}, appelle à restructurer les GT sans modifier leur composition telle que fixée par le décret 2022-446 du 20 Juillet 2022

\begin{figure}[c]
\includegraphics[height=6.0cm]{DOMAINES.PNG}
\end{figure}
\end{frame}

\subsection{\tiny Utilisation des terminologies appropriées}
\begin{frame}{Utilisation des terminologies appropriées}

\begin{enumerate} [<+->]
 \item  \alert{Etudes Nationales de Perspectives à LT (NLTPS)}, terminologie recommandée pour 2 raisons  \vfill
     \begin{itemize} [<+->]
     \item \small cohérence avec NLTPS Bénin 2025 (cf. méthodologie Futurs Africains);  \vfill
     \item \small nécessité d’un terme générique et globalisant (études, aspirations, diagnostic, scénarios…) \vfill
      \end{itemize}
  \item  \alert{Vision Bénin 2060} est l’image projetée par les Béninois. Il conviendrait alors de parler de : \vfill
     \begin{itemize} [<+->]
    \item \small Construction de la Vision au lieu de formulation de la Vision  \vfill
    \end{itemize}
\item \alert{Cadre Stratégique de Développement à Long Terme (CSD-LT)}, instrument de planification 
     \begin{itemize} [<+->]
    \item \small concrétisant la Vision, dans le temps, l’espace et au niveau des différents acteurs de dvpmt \vfill
    \item \small décliné en PNDs $\Rightarrow$  Plans opérationnels $\Rightarrow$ Programmes de dépenses $\Rightarrow$ Lois de finances… et \vfill
    \item \small servant de référentiel stratégique pour les ANE (secteur privé, société civile et PTF) \vfill
    \end{itemize}
\end{enumerate}
\end{frame}

\subsection{\tiny Expertise additionnelle en appui au processus}
\begin{frame}{Expertise additionnelle en appui au processus}
\begin{itemize} [<+->]
 \item  \textbf {Etat des lieux de l’expertise disponible}  \vfill
    \begin{itemize} [<+->]
   \item Personnes-ressources du CTP, ST/DGPD et des GT \vfill
   \item Cadres à différents niveaux de l’administration impliqués dans le processus \vfill
   \item  Taskforce de 3 experts \vfill
    \end{itemize}
 \item \textbf {Besoins d’expertise}  \vfill
    \begin{itemize} [<+->]
    \item Renforcement de l’effectif actuel de la Taskforce (économiste, institutionnaliste, spécialistes TIC, sécurité, santé, et éducation et personnel de soutien) suivant des modalités à définir   \vfill
    \item Consultants à recruter pour la réalisation des études (définir les modalités) \vfill
    \end{itemize}
\end{itemize}
\end{frame}

\section{Feuille de route révisée}
\subsection{\tiny Considérations générales}
\begin{frame}{Considérations générales}
  \begin{itemize} [<+->]
  \item Nous avons répertorié les activités par étape. Seules les activités principales sont mentionnées pour conserver la facilité de lecture et d’exploitation du document; \vfill
 \item  L’animation des organes du cadre institutionnel  est  transversale au processus;  \vfill
 \item  A chaque phase, sont prévus des ateliers de pré-validation et de validation des produits d’étape ;  ainsi qu’une activité de suivi à des fins de compte rendu au CTP;  \vfill
 \item  Les activités ne sont pas présentées dans un ordre chronologique, la programmation temporelle (figurant dans la colonne calendrier  présentée ci-dessous) indique clairement les séquences suivant lesquelles, elles sont réalisées.  \vfill
 \end{itemize}
\end{frame}

\subsection{\tiny Activités planifiées par étape du processus}
\begin{frame}{Activités planifiées par étape du processus}
  \begin{enumerate} [<+->]
  \item  L’animation des organes du cadre institutionnel  \vfill
 \item  l’évaluation de l’opérationnalisation de la Vision Bénin 2025 Alafia ; \vfill
 \item  le lancement du processus NLTPS Bénin 2060 ; \vfill
 \item  le cadrage opérationnel du processus ; \vfill
\item  le renforcement des capacités des acteurs en charge de la conduite du processus ; \vfill
\item  le cadrage méthodologique du processus ; \vfill
\item  l’identification des aspirations des populations béninoises ; \vfill
 \end{enumerate}
\end{frame}

\begin{frame}{Activités planifiées par étape du processus (suite)}
%\begin{enumerate[(i.)]
  \begin{enumerate} [<+->][8]
  \item   la réalisation des études prospectives dans les domaines retenus ;  \vfill
 \end{enumerate}
  \begin{enumerate} [<+->][9]
 \item   le diagnostic stratégique du système-Bénin ; \vfill
 \end{enumerate}
  \begin{enumerate} [<+->][10]
 \item  la construction des scénarios  \vfill
 \end{enumerate}
  \begin{enumerate} [<+->][11]
 \item le choix des stratégies ; \vfill
 \end{enumerate}
  \begin{enumerate} [<+->][12]
\item  les travaux de rédaction finale ; \vfill
 \end{enumerate}
  \begin{enumerate} [<+->][13]
\item  la pré-validation des documents NLTPS Bénin 2060 ; \vfill
 \end{enumerate}
  \begin{enumerate} [<+->][14]
\item  la validation des documents NLTPS Bénin 2060. \vfill
 \end{enumerate}
\end{frame}

\subsection{\tiny Produits finis des activités}
\begin{frame}{Produits finis des activités}
  \begin{itemize} [<+->]
  \item  Les produits finis des activités précisent dans un esprit de gestion axée sur les résultats, ce qui est attendu de la mise en œuvre de chaque activité \vfill
 \item Les produits peuvent être des comptes rendus de réunions, des notes techniques, les différentes versions de rapports  \vfill
 \end{itemize}
\end{frame}

\subsection{\tiny Calendrier et  délais de mise en oeuvre}
\begin{frame}{Calendrier et  délais de mise en oeuvre}
  \begin{itemize} [<+->]
  \item  La programmation dans le temps est faite sur une période de vingt (20) mois suite aux recommandations du Président du CTP et des contraintes de durée réaliste pour la mise en œuvre technique; \vfill
 \item Une échéance de mise en oeuvre est proposée aux différentes étapes du processus en s’inspirant de l’expérience des NLTPS Vision Bénin 2025 Alafia et des guides méthodologiques existants; \vfill
\item  Cette programmation donc est minimaliste et doit interpeler la disponibilité effective de toutes les parties prenantes; \vfill
\item   Bien qu’il existe une programmation dans le temps, la Taskforce a jugé utile de prévoir une colonne pour fixer les dates considérées comme butoirs \vfill
 \end{itemize}
\end{frame}

\subsection{\tiny Responsables et autres intervenants}
\begin{frame}{Responsables et autres intervenants}
  \begin{itemize} [<+->]
 \item  Les structures responsables de la réalisation de chaque activité programmée sont identifiées et litées dans la colonne des responsables; \vfill
 \item Référence au Décret N° 2022-446 du 20 juillet 2022, aux orientations du Président du CTP et  aux termes de référence des experts de la Taskforce \vfill
\item  La colonne  " autres intervenants " est prévue pour identifier et lister les autres personnes ou instances qui contribuent à la réalisation des activités. \vfill
 \end{itemize}
\end{frame}




\section{Programme de travail trimestriel et checklist}
\subsection{\tiny Programme de travail trimestriel }
\begin{frame}{Programme de travail trimestriel }
\textbf{Notre programme de travail est extrait de la feuille de route globale affinée}  \vfill 
\begin{itemize} [<+->]
\item  08 groupes d’activités sont prévues avec des livrables bien définis   \vfill
\item Certaines sont permanentes (les réunions du \textbf{CNOS, CTP ou STRP}) et d’autres sont ponctuelles.  \vfill
\item  Il y a des activités planifiées qui ont été exécutées et d’autres non exécutées et qui seront reportées sur le premier trimestre 2024  \vfill
\end{itemize}
\end{frame}

\begin{frame}{Programme de travail du bimestre Novembre-Décembre}
\begin{enumerate} [<+->]
\item \textbf{La Taskforce a exécuté les activités suivantes}
\begin{enumerate} [<+->]
\item La participation aux préparatifs de lancement du processus NLTPS Bénin 2060 \vfill
\item La participation au pré-cadrage du renforcement des capacités \vfill
\item La relecture et l’amendement de la note conceptuelle et de la feuille de route \vfill
\item L’élaboration d’une checklist de suivi des activités. \vfill
\item La revue de la documentation des NLTPS Bénin 2025 \vfill
\item Une revue documentaire sélective sur les exercices NLTPS et autres études prospectives \vfill
\end{enumerate}
%\item \textbf{Ces activités s’inscrivent dans la perspective \textbf{(i)} de la relecture de la note conceptuelle ; \textbf{(ii)} de l’élaboration du guide de réflexion prospective qui est l’un des produits attendus de la Taskforce.}
\end{enumerate}
\end{frame}


\begin{frame}{L'essentiel à retenir de la revue documentaire}

\begin{exampleblock}{  \textbf{ Les différentes étapes du processus} }
     \begin{enumerate} [<+->]
      \item   Analyse rétrospective  : Matrice de Diagnostic Stratégique  \vfill
      \item    Analyse structurelle : matrice d'analyse structurelle (MAS), \textbf{Application du MICMAC sur les données de NLTPS 2025} \vfill
      \item    Analyse du jeu des acteurs: MID Acteur par Acteur , Matrice Acteurs par Sous-système , \textbf{Application du MACTOR}  \vfill
     \end{enumerate}
\end{exampleblock}

\begin{block}{ \textbf{Evolution du système Bénin} }
  \begin{enumerate} [<+->]
  \item   Formulation des questions – clés et  définition des hypothèses d’évolution   \vfill
  \item   Construction des micro-scénarios ou scénarios thématiques : \textbf{Application du Scénaring Tools pour l'analyse morphologique emboitée}  \vfill
  \item   Construction des scénarios globaux : \textbf{diagramme de VEITCH} 
  \end{enumerate}
\end{block}
\end{frame}

\begin{frame}{ L'essentiel à retenir de la revue documentaire}
\begin{block}{ \textbf{Points d'attention sur la méthodologie de collecte} }
\begin{itemize} [<+->]
  \item   Reconsidérer les  groupes cibles et les critères de choix des différentes localités: Ratisser large ( ménage,  experts, leaders d'opinion, associations)     \vfill
  \item   Taille de l’échantillon de l’enquête par questionnaire : \textbf{sous-représentation des femmes (10,6 \%)}, \textbf{hétérogénéité des groupes de discussion}     \vfill
   \item Methodes modernes de traitement des données qualitatives : \textbf{Le NVIVO}
  \end{itemize}
\end{block}
\end{frame}

\begin{frame}{ Aspirations pendant le processus}
\begin{exampleblock}{ \textbf{format du processus} }
     \begin{enumerate}  [<+->]
      \item    Retrospectives, analyses structurelles, jeu des acteurs \vfill  
      \item    Construction des scénarios tendanciels: Outils  \vfill
      \item    Collecte des aspirations : Outils  \vfill
            \begin{itemize} [<+->]
            \item   Questionnaire:questions sur l'identification, la perception des populations sur la situation passée et présente  et les aspirations (microscénarios) \vfill
            \item   Guide d’entretien
            \item   Le Color Insight: mettre en œuvre la méthode abaque de consultation des experts
            \end{itemize} 
      \item Construction des scénarios normatifs (analyse morphologique emboitée  + aspirations des populations)
     \end{enumerate}
\end{exampleblock}
\end{frame}


\begin{frame}{Programme de travail trimestriel Janvier-Mars}
\textbf{Le programme de travail Janvier-Mars prévoit d’autres activités au nombre desquelles :}  \vfill 
\begin{itemize} [<+->]
\item  L’appui technique au renforcement des capacités des cadres impliqués dans le processus à travers la contribution de la Taskforce à l’identification des logiciels et des documents de prospective nécessaires à la conduite du processus;   \vfill
\item Des discussions sur la contribution technique de la Taskforce sur le positionnement de l’identification des aspirations dans le processus NLTPS Bénin 2060;  \vfill
\item Des discussions sur la contribution technique de la Taskforce portant sur le choix des domaines et des thématiques d’études  \vfill
\item L’appui technique de la Taskforce à la tenue des sessions des instances du cadre institutionnel attendues pour le mois de décembre conformément au décret \vfill
 \end{itemize}
\end{frame}

\subsection{\tiny Checklist de suivi du processus}
\begin{frame}{Checklist de suivi du processus}
\begin{itemize} [<+->]
\item La checklist est l’outil opérationnel proposé par la Taskforce pour faire le suivi de l’état d’avancement des activités de la feuille de route.  \vfill
\item Il est attendu que cet outil soit validé pour être utilisé pour les compte rendus, les rapports d’avancement du processus.\vfill
\item Une illustration de l’outil pour la période novembre 23 - mars 2024.  \vfill
\end{itemize}
\end{frame}


\begin{frame}{Checklist de suivi du processus}
\begin{figure}[t]
\includegraphics[height=7.4cm]{Checklist.PNG}
\end{figure}
\end{frame}

\begin{frame}
\begin{center}
\LARGE Merci de votre aimable attention !
\end{center}
\end{frame}

\begin{frame}
\titlepage
\end{frame}






\end{document}


